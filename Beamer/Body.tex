\Frame<\shownframes>[\shortpaperID]{\shorttitle}{
\ExperimentSlide% CMS, Atlas... \ExpSlide{\IGH[\DefaultExpSize]{\thedir/Logo.png}}%
%\reinsertTitle% if covered by Logo
\ontop{\theCitation}%
%%%%%%%%%%%%%%%%%%%%%%%%%%%%%%%%%%%
\topright<1>{\IGH{../pdf/Parti-Suffrages-Camembert-PVL.pdf}}%
%%%%%%%%%%%%%%%%%%%%%%%%%%%%%%%%%%%
\topleft<1>{\smallpage{0.3}{
Nous avons fait 75\% de nos voix sur nos listes, et 18\% sur des
listes sans d�nomination (caveat).
\skipx
Ensuite, 2.4\% chez le PLR et 1.5\% au PS.
\skipx
Les grands partis font proportionnellement plus de suffrages sur leurs listes.
}}
%%%%%%%%%%%%%%%%%%%%%%%%%%%%%%%%%%%
\topright<2>{\IG[0.5]{../pdf/Parti-Bilan-PVL.pdf}}%
\topleft<2-3>{\IG[0.5]{../pdf/Parti-Parasite-PVL.pdf}}%
\topright<3>{\IG[0.5]{../pdf/Parti-CandidatsParasite-PVL.pdf}}%
%%%%%%%%%%%%%%%%%%%%%%%%%%%%%%%%%%%
\ontheleft<2-3>[0.7]{\fullpage{
Les PLR font pr�s de 5000 voix sur nos listes, mais nous en faisons 6000 chez eux \Arrow Bilan positif. Nous perdons un peu au Centre.
\skipx
\only<3>{Mahaim et Broulis ne cartonnent pas chez nos �lecteurs.}
}}
%%%%%%%%%%%%%%%%%%%%%%%%%%%%%%%%%%%
\topleft<4>{\IG[0.6]{../pdf/Liste-Suffrages-Par-Candidat-VERT_LIBERAUX.pdf}}%
\topright<4>{\IGH[0.5]{../pdf/Liste-Suffrages-Par-Candidat-ENGAGES_POUR_DEMAIN.pdf}}%
\bottomright<4>{\IGH[0.5]{../pdf/Liste-Suffrages-Par-Candidat-JEUNES_VERT_LIBERAUX.pdf}}%
\ontheleft<4-5>[0.8]{\smallpage{0.6}{R�sultats individuels.\\
Note: La barre bleu fonc� est
la somme de toutes les contributions.}}
%%%%%%%%%%%%%%%%%%%%%%%%%%%%%%%%%%%
\topleft<5>{\IG[0.6]{../pdf/Liste-Suffrages-Par-Candidat-PLR.pdf}}%
\topright<5>{\IGH[0.5]{../pdf/Liste-Suffrages-Par-Candidat-PSV.pdf}}%
\bottomright<5>{\IGH[0.5]{../pdf/Liste-Suffrages-Par-Candidat-Les_Vert_e_s.pdf}}%
} % end frame
