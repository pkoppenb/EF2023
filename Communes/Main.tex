\documentclass[aspectratio=32]{beamer}
\mode<presentation>
% packages
\usepackage{BeamerUsualSuspects}
\SetColor{PVL}
\def\zoom{0}
\def\showOverlayNumber{0}
\usetheme{Amsterdam}
\def\conference{1}                                                      
\date{}
\author[Patrick Koppenburg]{Patrick~Koppenburg}
\institute[Nikhef]{Nikhef, Amsterdam}
\def\meetingurl{\indico{}}
\logo{\WhiteScaleBox{\PVLLogo[0.055]}}
\def\thetitlecolor{\textcommon} % change that
\def\contactdetails{\bf\scriptsize{\citationblue[\bluesky{koppenburg.ch}]~[\mailto{patrick@koppenburg.ch}]}}
\renewcommand{\TheLogo}{\href{https://www.koppenburg.ch}{\ExpLogo[0.12]{github.png}}\ \ \PVLLogo[0.12]}
%%%%%%%%%%%%%
\newcommand{\Commune}[3][NONE]{
  \IF{#1}{NONE}{\def\LL{#2}}{\def\LL{#1}}
  \def\ww{0.49}
\Frame[\shortpaperID-#2]{Partis du Centre: \LL}{
\ontop{\theCitation}%
\Body{
\IG[\ww]{../pdf/Commune-#2-Partis.pdf}%
\IG[\ww]{../pdf/Commune-#2-Centres-Candidats.pdf}%
}
\ontheright[0.7]{\SP{#3}}
\IF{\inBackup}{1}{\bottomright{\tiny\hyperlink{BACKUP-TOC}{\nolinkurl{[retour]}}}}{}
}
}
%%%%%%%%%%%%%%%%%%%%%%%%%%%%%%%%%%%%%%%%%%%%%%%%%%%%%%%%%%%%%%%%%%%%
\newcommand<>{\Resultats}[2][0.2]{\anywhere#3[c]{0.92}{-0.1}{\rotatebox[origin=c]{-25}{\NewTransparentBox[lightbox]{#1}{0.85}{\black\scriptsize\centering\bf #2}}}}
\def\GoodBad{\def\stamp{\Resultats{PVL\,$>5\%$\\ \red Centre\,$<5\%$}}}
\def\GoodGood{\def\stamp{\Resultats{PVL\,$>5\%$\\ Centre\,$>5\%$}}}
\def\BadGood{\def\stamp{\Resultats{\red PVL\,$<5\%$\\ \black Centre\,$>5\%$}}}
\def\BadBad{\def\stamp{\Resultats{\red PVL\,$<5\%$\\ Centre\,$<5\%$}}}
%%%%%%%%%%%%%%%%%%%%%%%%%%%%%%%%%%%%%%%%%%%%%%%%%%%%%%%%%%%%%%%%%%%%
\begin{document}
%%%%%%%%%%%%%%%%%%%%%%%%%%%%%%%%%%%%%%%%%%%%%%%%%%%%%%%%%%%%%%%%%%%%
\InputFrame[1]{.}{Title}
%\InputFrame[1]{.}{Celine}
%\InputFrame[1]{.}{Uniques}
\InputFrame{.}{Communes}
\InputFrame{.}{Correlations}
\InputFrame{.}{Vaud}
\GoodBad
\Commune[Canton de Vaud]{Canton_de_Vaud}{Dans le canton, le PVL fait 7.5\%, le Centre 4.5\% et les Libres 1.2\%. La moiti� de ces suffrages viennent de listes compactes et l'autre de listes modifi�es.\skipx
  Si je ne regarde que les bulletins modifi�s, C�line arrive en t�te devant Fran�ois, puis suivent trois centres et un libre!}
%%%%%%%%%%%%%%%%%%%%%%%%
%
% Nous au-dessus du quorum, pas eux.
%
%\InputFrame[4]{.}{PVLCentre}
%
\def\stamp{
  \Resultats{PVL\,$>5\%$\\ \red Centre\,$<5\%$}%
  \anywhere<2->[c]{0.35}{0.25}{\NewBox{0.25}{\black On passe le quorum, pas le Centre}}
  \anywhere<2->[c]{0.35}{0.55}{\NewBox[redbox]{0.25}{\warningsign\ National $\neq$ Communal}}
  \anywhere<3->[c]{0.9}{0.25}{\NewBox[midbox]{0.25}{\black Qui sont les candidats populaires?}}
  \anywhere<3->[c]{0.85}{0.55}{\NewBox[redbox]{0.3}{\warningsign\  Ce ne sont que rarement des locaux}}
  \anywhere<3->[c]{0.7}{0.9}{\NewBox[redbox]{0.6}{\warningsign\ Il y a plus de listes modifi�es PVL que Centre \Arrow �a favorise nos candidats.}}
  \anywhere<4>[c]{0.5}{0.5}{\rotatebox{20}{\NewBox[textbox]{0.5}{Prenez ces chiffres avec pr�caution. C'est votre expertise locale qui compte.}}}
}
\Commune{Lausanne}{A Lausanne on est proportionnellement plus forts que le Centre. Leurs candidats font de mauvais scores individuels.\skipx
  La barre verticales indique le quorum � 5\%.
  \skipx
  Cas d'�cole: liste PVL ou liste commune?}
\GoodBad
\Commune{Pully}{Similaire � Lausanne.
  \skipx
  \hyperlink{BACKUP-Beamer-Lutry}{[Lutry]},
  \hyperlink{BACKUP-Beamer-Epalinges}{[Epalinges]}, et
  \hyperlink{BACKUP-Beamer-Le_Mont-sur-Lausanne}{[Le Mont]} sont similaires
} %  	5590, # 	Lavaux-Oron 	18 984 	5,85 	3245
%\GoodBad
%\Commune{Lutry}{} %  	5606, # 	Lavaux-Oron 	10 718 	8,45 	1268
\GoodBad
\Commune{Belmont-sur-Lausanne}{On m'a demand� Belmont. Le Centre est tr�s pr�s du quorum.} %
%\GoodBad
%\Commune{Rolle}{Rolle est dans une situation similaire.} %  	5861, # 	Nyon 	6 322 	2,74 	2307
\GoodBad
\Commune{Bussigny}{Exemple de l'Ouest Lausannois} %  	5624, # 	Ouest lausannois 	10 365 	4,82 	2150
%\GoodBad
%\Commune[�palinges]{Epalinges}{} %  	5584, # 	Lausanne 	9 822 	4,57 	2149
%\GoodBad
%\Commune[Le Mont-sur-Lausanne]{Le_Mont-sur-Lausanne}{} %  	5587, # 	Lausanne 	9 291 	9,82 	946
\GoodBad
\Commune{Morges}{Pr�fecture de Morges.} %  	5642, # 	Morges 	17 529 	3,85 	4553
\GoodBad
%\Commune{Montreux}{A Montreux ce sont les libres qui sortent du lot.} %  	5886, # 	Riviera-Pays-d'Enhaut 	26 230 	33,37 	786
\GoodBad
\Commune{Montreux}{Les candidats libres sont populaires. La liste ne l'est pas.\skipx
  \hyperlink{BACKUP-Beamer-Vevey}{[Vevey]} est similaire. } %  	5890, # 	Riviera-Pays-d'Enhaut 	19 738 	2,38 	8293

\GoodBad
\Commune{Bex}{Exemple du district d'Aigle} %  	5402, # 	Aigle 	8 167 	96,56 	85
%%%%%%%%%%%%%%%%%%%%%%%%%%%%%%%%%%%%%%%%%%%%%%%%%%%%%%%%%%%%%%%%%%%%%%%%
%
% Les deux en dessus
%
\InputFrame[1]{.}{PVLCentre}
\GoodGood
\Commune{Echallens}{Exemple du Gros-de-Vaud}
\GoodGood
\Commune{Nyon}{On fait pr�s de 10\% � Nyon, et ce sont les locaux (non candidats aux communales) qui tirent la liste.
  \\ ~ \\
  Plus grosse commune o� PVL et Centre passent le quorum.} %  	5724, # 	Nyon 	22 465 	6,79 	330
%
% Les deux en dessous
%
\InputFrame[2]{.}{PVLCentre}
\BadBad
\Commune{Yverdon-les-Bains}{A Yverdon le Centre est fort (mais moins que nous) et leurs candidats populaires.\\ %\skipx 
  Plus grosse commune o� ni PVL ni Centre ne passent le quorum.\skipx
  \hyperlink{BACKUP-Beamer-Renens_VD_}{[Renens]} et
  \hyperlink{BACKUP-Beamer-Chavannes-pres-Renens}{[Chavannes-pr�s-Renens]} sont similaires.
} %  	5938, # 	Jura-Nord vaudois 	29 827 	11,28 	2644
%\BadBad
%\Commune[Renens]{Renens_VD_}{Le PVL et le Centre sont �galement faibles.} %  	5591, # 	Ouest lausannois 	21 086 	2,96 	7124
%\Commune{Gland}{Gland, c'est sp�cial} %  	5721, # 	Nyon 	13 664 	8,32 	1642
%\Commune[�cublens]{Ecublens_VD_}{} %  	5635, # 	Ouest lausannois 	13 118 	5,71 	2297
%\Commune[La Tour-de-Peilz]{La_Tour-de-Peilz}{} %  	5889, # 	Riviera-Pays-d'Enhaut 	12 382 	3,24 	3822
%\BadBad
%\Commune[Chavannes-pr�s-Renens]{Chavannes-pres-Renens}{} %  	5627, # 	Ouest lausannois 	8 737 	1,65 	5295
%\BadBad
%\Commune{Prilly}{Le Centre est plus fort.} %  	5589, # 	Ouest lausannois 	12 294 	2,19 	5614
%\BadBad
%\Commune{Aigle}{Attention on quorum. Et Michael ne tirera pas la liste.} %  	5401 , #	Aigle 	10 913 	16,41 	665
\BadBad
\Commune{Payerne}{On est derri�re le Centre!
  \skipx
  \hyperlink{BACKUP-Beamer-Prilly}{[Prilly]} est similaire.
} %  	5822, # 	Broye-Vully 	10 342 	24,19 	428

%\Commune[Blonay Saint-L�gier]{Blonay_-_Saint-Legier}{} % 	5892, # 	Riviera-Pays-d'Enhaut 	12 117 	31,24 	388
%\Commune{Crissier}{} %  	5583, # 	Ouest lausannois 	9 181 	5,51 	1666
%\Commune{Ollon}{} %  	5409, # 	Aigle 	7 968 	59,56 	134
%\Commune{Orbe}{} %  	5757, # 	Jura-Nord vaudois 	7 613 	12,02 	633
%\Commune{Moudon}{} %  	5678, # 	Broye-Vully 	6 124 	15,69 	390
%\Commune{Oron}{} %

\InputFrame[3]{.}{PVLCentre}
\BadGood
\Commune{Valbroye}{ (Granges-Marnand) Plus grosse commune o� le Centre passe le quorum mais pas le PVL. 3300 habitants.} %  	5822, # 	Broye-Vully 	10 342 	24,19 	428

%%%%%%%%%%%%%%%%%%%%%%%%
\def\aVeil{\SlideVeil<2>[0.7]}
\Conclusion[LeCentre.png]{\black
  \Enumerate<2>{\black Il y a de nombreuses villes o� nous passons le quorum, mais pas le Centre
    \Itemize[ \Arrow ]{\black La question d'une liste commune se pose (surtout pour le Centre et les Libres).
    \item[\notok] Mais attention aux locomotives du Centre ou Libres qui peuvent prendre les si�ges.}
  \item Si PVL et Centre passent largement le quorum \Arrow deux listes apparent�es
  \item Si aucun ne passe le quorum  \Arrow une liste
  \item {\bf Ce sont l� des consid�rations statistiques. La politique locale prime.}
  }
  \only<2>{\skipx[2]
  Le code, un document de 290 pages, 3000 graphiques, et ces transparents sont � \showhref{https://github.com/pkoppenb/EF2023/}}.
}
\Backup
%%%%%%%%%%%%%%%%%%%%%%%%
\InputFrame{.}{TOC}
% ls -1 ../pdf/Commune-*-Partis.pdf | awk -F"Commune-" '{print $2}' | awk -F"-Partis" '{print $1}' 

\Commune[Aclens]{Aclens}{}
\Commune[Agiez]{Agiez}{}
\Commune[Aigle]{Aigle}{}
\Commune[Allaman]{Allaman}{}
\Commune[Arnex-sur-Nyon]{Arnex-sur-Nyon}{}
\Commune[Arnex-sur-Orbe]{Arnex-sur-Orbe}{}
\Commune[Arzier-Le Muids]{Arzier-Le_Muids}{}
\Commune[Assens]{Assens}{}
\Commune[Aubonne]{Aubonne}{}
\Commune[Avenches]{Avenches}{}
\Commune[Ballaigues]{Ballaigues}{}
\Commune[Ballens]{Ballens}{}
\Commune[Bassins]{Bassins}{}
\Commune[Baulmes]{Baulmes}{}
\Commune[Bavois]{Bavois}{}
\Commune[Begnins]{Begnins}{}
\Commune[Belmont-sur-Lausanne]{Belmont-sur-Lausanne}{}
\Commune[Belmont-sur-Yverdon]{Belmont-sur-Yverdon}{}
\Commune[Bercher]{Bercher}{}
\Commune[Berolle]{Berolle}{}
\Commune[Bettens]{Bettens}{}
\Commune[Bex]{Bex}{}
\Commune[Biere]{Biere}{}
\Commune[Bioley-Magnoux]{Bioley-Magnoux}{}
\Commune[Blonay - Saint-Legier]{Blonay_-_Saint-Legier}{}
\Commune[Bofflens]{Bofflens}{}
\Commune[Bogis-Bossey]{Bogis-Bossey}{}
\Commune[Bonvillars]{Bonvillars}{}
\Commune[Borex]{Borex}{}
\Commune[Bottens]{Bottens}{}
\Commune[Bougy-Villars]{Bougy-Villars}{}
\Commune[Boulens]{Boulens}{}
\Commune[Bourg-en-Lavaux]{Bourg-en-Lavaux}{}
\Commune[Bournens]{Bournens}{}
\Commune[Boussens]{Boussens}{}
\Commune[Bremblens]{Bremblens}{}
\Commune[Bretigny-sur-Morrens]{Bretigny-sur-Morrens}{}
\Commune[Bretonnieres]{Bretonnieres}{}
\Commune[Buchillon]{Buchillon}{}
\Commune[Bullet]{Bullet}{}
\Commune[Bursinel]{Bursinel}{}
\Commune[Bursins]{Bursins}{}
\Commune[Burtigny]{Burtigny}{}
\Commune[Bussigny]{Bussigny}{}
\Commune[Bussy-sur-Moudon]{Bussy-sur-Moudon}{}
%\Commune[Canton de Vaud]{Canton_de_Vaud}{}
\Commune[Chamblon]{Chamblon}{}
\Commune[Champagne]{Champagne}{}
\Commune[Champtauroz]{Champtauroz}{}
\Commune[Champvent]{Champvent}{}
\Commune[Chardonne]{Chardonne}{}
\Commune[Chateau-d Oex]{Chateau-d_Oex}{}
\Commune[Chavannes-de-Bogis]{Chavannes-de-Bogis}{}
\Commune[Chavannes-des-Bois]{Chavannes-des-Bois}{}
\Commune[Chavannes-le-Chene]{Chavannes-le-Chene}{}
\Commune[Chavannes-le-Veyron]{Chavannes-le-Veyron}{}
\Commune[Chavannes-pres-Renens]{Chavannes-pres-Renens}{}
\Commune[Chavannes-sur-Moudon]{Chavannes-sur-Moudon}{}
\Commune[Chavornay]{Chavornay}{}
\Commune[Chene-Paquier]{Chene-Paquier}{}
\Commune[Cheseaux-Noreaz]{Cheseaux-Noreaz}{}
\Commune[Cheseaux-sur-Lausanne]{Cheseaux-sur-Lausanne}{}
\Commune[Cheserex]{Cheserex}{}
\Commune[Chessel]{Chessel}{}
\Commune[Chevilly]{Chevilly}{}
\Commune[Chevroux]{Chevroux}{}
\Commune[Chexbres]{Chexbres}{}
\Commune[Chigny]{Chigny}{}
\Commune[Clarmont]{Clarmont}{}
\Commune[Coinsins]{Coinsins}{}
\Commune[Commugny]{Commugny}{}
\Commune[Concise]{Concise}{}
\Commune[Coppet]{Coppet}{}
\Commune[Corbeyrier]{Corbeyrier}{}
\Commune[Corcelles-le-Jorat]{Corcelles-le-Jorat}{}
\Commune[Corcelles-pres-Concise]{Corcelles-pres-Concise}{}
\Commune[Corcelles-pres-Payerne]{Corcelles-pres-Payerne}{}
\Commune[Corseaux]{Corseaux}{}
\Commune[Corsier-sur-Vevey]{Corsier-sur-Vevey}{}
\Commune[Cossonay]{Cossonay}{}
\Commune[Crans]{Crans_VD_}{}
\Commune[Crassier]{Crassier}{}
\Commune[Crissier]{Crissier}{}
\Commune[Cronay]{Cronay}{}
\Commune[Croy]{Croy}{}
\Commune[Cuarnens]{Cuarnens}{}
\Commune[Cuarny]{Cuarny}{}
\Commune[Cudrefin]{Cudrefin}{}
\Commune[Cugy]{Cugy_VD_}{}
\Commune[Curtilles]{Curtilles}{}
\Commune[Daillens]{Daillens}{}
\Commune[Demoret]{Demoret}{}
\Commune[Denens]{Denens}{}
\Commune[Denges]{Denges}{}
\Commune[Dizy]{Dizy}{}
\Commune[Dompierre]{Dompierre_VD_}{}
\Commune[Donneloye]{Donneloye}{}
\Commune[Duillier]{Duillier}{}
\Commune[Dully]{Dully}{}
\Commune[Echallens]{Echallens}{}
\Commune[Echandens]{Echandens}{}
\Commune[Echichens]{Echichens}{}
\Commune[Eclepens]{Eclepens}{}
\Commune[Ecublens]{Ecublens_VD_}{}
\Commune[Epalinges]{Epalinges}{}
\Commune[Ependes]{Ependes_VD_}{}
\Commune[Essertines-sur-Rolle]{Essertines-sur-Rolle}{}
\Commune[Essertines-sur-Yverdon]{Essertines-sur-Yverdon}{}
\Commune[Etagnieres]{Etagnieres}{}
\Commune[Etoy]{Etoy}{}
\Commune[Eysins]{Eysins}{}
\Commune[Faoug]{Faoug}{}
\Commune[Fechy]{Fechy}{}
\Commune[Ferreyres]{Ferreyres}{}
\Commune[Fey]{Fey}{}
\Commune[Fiez]{Fiez}{}
\Commune[Fontaines-sur-Grandson]{Fontaines-sur-Grandson}{}
\Commune[Forel Lavaux ]{Forel_Lavaux_}{}
\Commune[Founex]{Founex}{}
\Commune[Froideville]{Froideville}{}
\Commune[Genolier]{Genolier}{}
\Commune[Giez]{Giez}{}
\Commune[Gilly]{Gilly}{}
\Commune[Gimel]{Gimel}{}
\Commune[Gingins]{Gingins}{}
\Commune[Givrins]{Givrins}{}
\Commune[Gland]{Gland}{}
\Commune[Gollion]{Gollion}{}
\Commune[Goumoens]{Goumoens}{}
\Commune[Grancy]{Grancy}{}
\Commune[Grandcour]{Grandcour}{}
\Commune[Grandevent]{Grandevent}{}
\Commune[Grandson]{Grandson}{}
\Commune[Grens]{Grens}{}
\Commune[Gryon]{Gryon}{}
\Commune[Hautemorges]{Hautemorges}{}
\Commune[Henniez]{Henniez}{}
\Commune[Hermenches]{Hermenches}{}
\Commune[Jongny]{Jongny}{}
\Commune[Jorat-Menthue]{Jorat-Menthue}{}
\Commune[Jorat-Mezieres]{Jorat-Mezieres}{}
\Commune[Jouxtens-Mezery]{Jouxtens-Mezery}{}
\Commune[Juriens]{Juriens}{}
\Commune[L'Abbaye]{L_Abbaye}{}
\Commune[L'Abergement]{L_Abergement}{}
\Commune[La Chaux Cossonay ]{La_Chaux_Cossonay_}{}
\Commune[La Praz]{La_Praz}{}
\Commune[La Rippe]{La_Rippe}{}
\Commune[La Sarraz]{La_Sarraz}{}
\Commune[La Tour-de-Peilz]{La_Tour-de-Peilz}{}
\Commune[Lausanne]{Lausanne}{}
\Commune[Lavey-Morcles]{Lavey-Morcles}{}
\Commune[Lavigny]{Lavigny}{}
\Commune[Le Chenit]{Le_Chenit}{}
\Commune[Le Lieu]{Le_Lieu}{}
\Commune[Le Mont-sur-Lausanne]{Le_Mont-sur-Lausanne}{}
\Commune[Les Clees]{Les_Clees}{}
\Commune[Le Vaud]{Le_Vaud}{}
\Commune[Leysin]{Leysin}{}
\Commune[Lignerolle]{Lignerolle}{}
\Commune[L'Isle]{L_Isle}{}
\Commune[Lonay]{Lonay}{}
\Commune[Longirod]{Longirod}{}
\Commune[Lovatens]{Lovatens}{}
\Commune[Lucens]{Lucens}{}
\Commune[Luins]{Luins}{}
\Commune[Lully]{Lully_VD_}{}
\Commune[Lussery-Villars]{Lussery-Villars}{}
\Commune[Lussy-sur-Morges]{Lussy-sur-Morges}{}
\Commune[Lutry]{Lutry}{}
\Commune[Maracon]{Maracon}{}
\Commune[Marchissy]{Marchissy}{}
\Commune[Mathod]{Mathod}{}
\Commune[Mauborget]{Mauborget}{}
\Commune[Mauraz]{Mauraz}{}
\Commune[Mex]{Mex_VD_}{}
\Commune[Mies]{Mies}{}
\Commune[Missy]{Missy}{}
\Commune[Moiry]{Moiry}{}
\Commune[Mollens]{Mollens_VD_}{}
\Commune[Molondin]{Molondin}{}
\Commune[Montagny-pres-Yverdon]{Montagny-pres-Yverdon}{}
\Commune[Montanaire]{Montanaire}{}
\Commune[Montcherand]{Montcherand}{}
\Commune[Montilliez]{Montilliez}{}
\Commune[Mont-la-Ville]{Mont-la-Ville}{}
\Commune[Montpreveyres]{Montpreveyres}{}
\Commune[Montreux]{Montreux}{}
\Commune[Montricher]{Montricher}{}
\Commune[Mont-sur-Rolle]{Mont-sur-Rolle}{}
\Commune[Morges]{Morges}{}
\Commune[Morrens]{Morrens_VD_}{}
\Commune[Moudon]{Moudon}{}
\Commune[Mutrux]{Mutrux}{}
\Commune[Novalles]{Novalles}{}
\Commune[Noville]{Noville}{}
\Commune[Nyon]{Nyon}{}
\Commune[Ogens]{Ogens}{}
\Commune[Ollon]{Ollon}{}
\Commune[Onnens]{Onnens_VD_}{}
\Commune[Oppens]{Oppens}{}
\Commune[Orbe]{Orbe}{}
\Commune[Orges]{Orges}{}
\Commune[Ormont-Dessous]{Ormont-Dessous}{}
\Commune[Ormont-Dessus]{Ormont-Dessus}{}
\Commune[Orny]{Orny}{}
\Commune[Oron]{Oron}{}
\Commune[Orzens]{Orzens}{}
\Commune[Oulens-sous-Echallens]{Oulens-sous-Echallens}{}
\Commune[Pailly]{Pailly}{}
\Commune[Paudex]{Paudex}{}
\Commune[Payerne]{Payerne}{}
\Commune[Penthalaz]{Penthalaz}{}
\Commune[Penthaz]{Penthaz}{}
\Commune[Penthereaz]{Penthereaz}{}
\Commune[Perroy]{Perroy}{}
\Commune[Poliez-Pittet]{Poliez-Pittet}{}
\Commune[Pompaples]{Pompaples}{}
\Commune[Pomy]{Pomy}{}
\Commune[Prangins]{Prangins}{}
\Commune[Premier]{Premier}{}
\Commune[Preverenges]{Preverenges}{}
\Commune[Prevonloup]{Prevonloup}{}
\Commune[Prilly]{Prilly}{}
\Commune[Provence]{Provence}{}
\Commune[Puidoux]{Puidoux}{}
\Commune[Pully]{Pully}{}
\Commune[Rances]{Rances}{}
\Commune[Renens]{Renens_VD_}{}
\Commune[Rennaz]{Rennaz}{}
\Commune[Rivaz]{Rivaz}{}
\Commune[Roche]{Roche_VD_}{}
\Commune[Rolle]{Rolle}{}
\Commune[Romainmotier-Envy]{Romainmotier-Envy}{}
\Commune[Romanel-sur-Lausanne]{Romanel-sur-Lausanne}{}
\Commune[Romanel-sur-Morges]{Romanel-sur-Morges}{}
\Commune[Ropraz]{Ropraz}{}
\Commune[Rossenges]{Rossenges}{}
\Commune[Rossiniere]{Rossiniere}{}
\Commune[Rougemont]{Rougemont}{}
\Commune[Rovray]{Rovray}{}
\Commune[Rueyres]{Rueyres}{}
\Commune[Saint-Barthelemy]{Saint-Barthelemy_VD_}{}
\Commune[Saint-Cergue]{Saint-Cergue}{}
\Commune[Sainte-Croix]{Sainte-Croix}{}
\Commune[Saint-George]{Saint-George}{}
\Commune[Saint-Livres]{Saint-Livres}{}
\Commune[Saint-Oyens]{Saint-Oyens}{}
\Commune[Saint-Prex]{Saint-Prex}{}
\Commune[Saint-Saphorin Lavaux ]{Saint-Saphorin_Lavaux_}{}
\Commune[Saint-Sulpice]{Saint-Sulpice_VD_}{}
\Commune[Saubraz]{Saubraz}{}
\Commune[Savigny]{Savigny}{}
\Commune[Senarclens]{Senarclens}{}
\Commune[Sergey]{Sergey}{}
\Commune[Servion]{Servion}{}
\Commune[Signy-Avenex]{Signy-Avenex}{}
\Commune[Suchy]{Suchy}{}
\Commune[Sullens]{Sullens}{}
\Commune[Suscevaz]{Suscevaz}{}
\Commune[Syens]{Syens}{}
\Commune[Tannay]{Tannay}{}
\Commune[Tartegnin]{Tartegnin}{}
\Commune[Tevenon]{Tevenon}{}
\Commune[Tolochenaz]{Tolochenaz}{}
\Commune[Trelex]{Trelex}{}
\Commune[Treycovagnes]{Treycovagnes}{}
\Commune[Trey]{Trey}{}
\Commune[Treytorrens Payerne ]{Treytorrens_Payerne_}{}
\Commune[Ursins]{Ursins}{}
\Commune[Valbroye]{Valbroye}{}
\Commune[Valeyres-sous-Montagny]{Valeyres-sous-Montagny}{}
\Commune[Valeyres-sous-Rances]{Valeyres-sous-Rances}{}
\Commune[Valeyres-sous-Ursins]{Valeyres-sous-Ursins}{}
\Commune[Vallorbe]{Vallorbe}{}
\Commune[Vaulion]{Vaulion}{}
\Commune[Vaux-sur-Morges]{Vaux-sur-Morges}{}
\Commune[VD-CH de l etranger]{VD-CH_de_l_etranger}{}
\Commune[Vevey]{Vevey}{}
\Commune[Veytaux]{Veytaux}{}
\Commune[Vich]{Vich}{}
\Commune[Villars-Epeney]{Villars-Epeney}{}
\Commune[Villars-le-Comte]{Villars-le-Comte}{}
\Commune[Villars-le-Terroir]{Villars-le-Terroir}{}
\Commune[Villars-Sainte-Croix]{Villars-Sainte-Croix}{}
\Commune[Villars-sous-Yens]{Villars-sous-Yens}{}
\Commune[Villarzel]{Villarzel}{}
\Commune[Villeneuve]{Villeneuve_VD_}{}
\Commune[Vinzel]{Vinzel}{}
\Commune[Vuarrens]{Vuarrens}{}
\Commune[Vucherens]{Vucherens}{}
\Commune[Vufflens-la-Ville]{Vufflens-la-Ville}{}
\Commune[Vufflens-le-Chateau]{Vufflens-le-Chateau}{}
\Commune[Vugelles-La Mothe]{Vugelles-La_Mothe}{}
\Commune[Vuiteboeuf]{Vuiteboeuf}{}
\Commune[Vulliens]{Vulliens}{}
\Commune[Vullierens]{Vullierens}{}
\Commune[Vully-les-Lacs]{Vully-les-Lacs}{}
\Commune[Yens]{Yens}{}
\Commune[Yverdon-les-Bains]{Yverdon-les-Bains}{}
\Commune[Yvonand]{Yvonand}{}
\Commune[Yvorne]{Yvorne}{}


%%%%%%%%%%%%%%%%%%%%%%%
\end{document}
