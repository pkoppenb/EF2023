\documentclass[aspectratio=32]{beamer}
\mode<presentation>
% packages
\usepackage{BeamerUsualSuspects}
\SetColor{PVL}
\def\zoom{1}
\def\showOverlayNumber{0}
\usetheme{Amsterdam}
\def\conference{1}                                                      
\date{}
\author[Patrick Koppenburg]{Patrick~Koppenburg}
\institute[Nikhef]{Nikhef, Amsterdam}
\def\meetingurl{\indico{}}
\logo{\WhiteScaleBox{\PVLLogo[0.055]}}
\def\thetitlecolor{\textcommon} % change that
\def\contactdetails{\bf\scriptsize{\citationblue[\bluesky{pkoppenburg.bsky.social}]~\citationblue[\twitter{pkoppenburg}]~[\mailto{patrick@koppenburg.ch}]}}
\renewcommand{\TheLogo}{\href{https://www.koppenburg.ch}{\ExpLogo[0.12]{QR-koppenburg.pdf}}\ \ \PVLLogo[0.12]}
%%%%%%%%%%%%%%%%%%%%%%%%%%%%%%%%%%%%%%%%%%%%%%%%%%%%%%%%%%%%%%%%%%%%
\begin{document}
%%%%%%%%%%%%%%%%%%%%%%%%%%%%%%%%%%%%%%%%%%%%%%%%%%%%%%%%%%%%%%%%%%%%
\InputFrame[1]{.}{Title}
\InputBody[1-4]{.}
\InputFrame[1]{.}{Celine}
\InputFrame[1]{.}{Uniques}
\InputFrame[5]{.}{Commune}
\InputFrame{.}{Communes}
\Conclusion{\black
  \Enumerate{\black On va chercher plus de voix chez les autres partis qu'on n'en perd.
    \Itemize[ \MidArrow ]{\black Il faut aller chercher ces �lecteurs.}
  \item J'esp�re que ces graphiques serviront aux sections pour �tablir des listes
    lors des prochaines �lections cantonales.
  }
  \skipx[2]
  Le code, un document de 250 pages, 2600 graphiques, et ces transparents sont � \showhref{https://github.com/pkoppenb/EF2023/blob/main/README.md}.
}
\Backup
%\Input{AllPlots}
\InputFrame{.}{Voc}
\InputBody[5-]{.} % Doublages
\InputFrame{.}{Matrice}
%\InputFrame[2]{.}{Celine}
\InputFrame{.}{Liste}
\InputFrame{.}{Parti}
\InputFrame{.}{Candidat}
\InputFrame{.}{Commune}
\end{document}
