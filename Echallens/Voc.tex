\Frame<\shownframes>[\shortpaperID]{Vocabulaire}{
\ExperimentSlide% CMS, Atlas... \ExpSlide{\IGH[\DefaultExpSize]{\thedir/Logo.png}}%
%\reinsertTitle% if covered by Logo
\ontop{\theCitation}%
\Body[1.3]{
%%%%%%%%%%%%%%%%%%%%%%%%%%%%%%%%%%%
\begin{description}[\quad]
\item[Un suffrage] est un vote exprim� par l'�lecteur. Au Canton de Vaud chaque �lecteur a 19 suffrages.
\item[Un bulletin compact] est une liste que l'�lecteur a mise dans l'urne sans rien y changer. Cela donne
19 suffrages � la liste.
\item[Un bulletin modifi�] est tout bulletin valable qui n'est pas compact. L'�lecteur peut avoir biff� des noms, ajout� d'autres, et doubl� des candidats.
\item[Un bulletin sans d�nomination] est le bulletin vierge sur lequel l'�lecteur a compos� sa propre liste de candidats. A noter que si l'�lecteur a y inscrit un nom de parti, cela devient un bulletin modifi� et figure pas dans la cat�gorie ci-dessus.\\
Je me suis permis une petite entorse: si un bulletin sans d�nomination comporte 10 ou plus candidats du m�me parti,\footnote{C'est bien 10 du m�me parti. Je choisis ensuite la liste comme celle qui a re�u le plus de suffrages.} je le compte comme bulletin modifi� et non comme sans d�nomination.
\item[Un suffrage suppl�mentaire] va � une liste mais pas � un candidat. Cela advient pour les listes incompl�tes (UDF par exemple), si un �lecteur biffe des candidats sans les remplacer, ou s'il met un nom de parti sur le bulletin vierge sans y mettre 19 candidats. 
\end{description}
}
%%%%%%%%%%%%%%%%%%%%%%%%%%%%%%%%%%%
} % end frame
